\documentclass[11pt,a4paper]{article} 

% PACKAGES FOR TITLES
\usepackage{titlesec}
\usepackage{color}

% PACKAGES FOR LANGUAGE AND FONT
\usepackage[utf8]{inputenc}
\usepackage[english]{babel}
\usepackage[T1]{fontenc} % Font encoding

% PACKAGES FOR IMAGES
\usepackage{graphicx}
\graphicspath{{img/}}
\usepackage{eso-pic} % For the background picture on the title page
\usepackage{subfig} % Numbered and caption subfigures using \subfloat
\usepackage{caption} % Coloured captions
\usepackage{transparent}

% STANDARD MATH PACKAGES
\usepackage{amsmath}
\usepackage{amsthm}
\usepackage{bm}
\usepackage[overload]{empheq}  % For braced-style systems of equations

% PACKAGES FOR TABLES
\usepackage{tabularx}
\usepackage{longtable} % tables that can span several pages
\usepackage{colortbl}

% PACKAGES FOR ALGORITHMS (PSEUDO-CODE)
\usepackage{algorithm}
\usepackage{algorithmic}

% PACKAGES FOR REFERENCES & BIBLIOGRAPHY
\usepackage[colorlinks=true,linkcolor=black,anchorcolor=black,citecolor=black,filecolor=black,menucolor=black,runcolor=black,urlcolor=black]{hyperref} % Adds clickable links at references
\usepackage{cleveref}
\usepackage[square, numbers, sort&compress]{natbib} % Square brackets, citing references with numbers, citations sorted by appearance in the text and compressed
\bibliographystyle{plain} % You may use a different style adapted to your field

% PACKAGES FOR THE APPENDIX
\usepackage{appendix}

% PACKAGES FOR ITEMIZE & ENUMERATES 
\usepackage{enumitem}

% OTHER PACKAGES
\usepackage{amsthm,thmtools,xcolor} % Coloured "Theorem"
\usepackage{comment} % Comment part of code
\usepackage{fancyhdr} % Fancy headers and footers
\usepackage{lipsum} % Insert dummy text
\usepackage{tcolorbox} % Create coloured boxes (e.g. the one for the key-words)

\newcommand{\bea}{\begin{eqnarray}} % Shortcut for equation arrays
\newcommand{\eea}{\end{eqnarray}}
\newcommand{\e}[1]{\times 10^{#1}}  % Powers of 10 notation
\newcommand{\mathbbm}[1]{\text{\usefont{U}{bbm}{m}{n}#1}} % From mathbbm.sty
\newcommand{\pdev}[2]{\frac{\partial#1}{\partial#2}}

\input{config_files/config}

\renewcommand{\title}{Title}
\newcommand{\AUTHORa}{Giuseppe Chiari}
\newcommand{\IDa}{10576799}
\newcommand{\AUTHORb}{Leonardo Gargani}
\newcommand{\IDb}{10569221}
\newcommand{\AUTHORc}{Serena Salvi}
\newcommand{\IDc}{10607377}
\newcommand{\course}{Computer Science and Engineering}
\newcommand{\supervisor}{Luca Bascetta}
\newcommand{\YEAR}{2022-2023}
\renewcommand{\abstract}{
...
}

\begin{document}

\AddToShipoutPicture*{\BackgroundPic}

\hspace{-0.6cm}\includegraphics[width=0.6\textwidth]{logo_polimi_ing_indinf.eps}

\vspace{-1mm}
\Large{\textbf{\color{bluePoli}{\title}}}\\

\vspace{-0.2cm}
\fontsize{0.3cm}{0.5cm}\selectfont \bfseries \textsc{\color{bluePoli} Project for the Control of Mobile Robots course \\ \course}\\

\vspace{-0.2cm}
\large{\textbf{\AUTHORa, \IDa}}\\
\large{\textbf{\AUTHORb, \IDb}}\\
\large{\textbf{\AUTHORc, \IDc}}

\small \normalfont

\vspace{11pt}

\centerline{\rule{1.0\textwidth}{0.4pt}}

\begin{center}
\begin{minipage}[t]{.24\textwidth}
\begin{minipage}{.90\textwidth}
\noindent
\scriptsize{\textbf{Supervisor:}} \\
\supervisor \\
\\
\textbf{Academic year:} \\
\YEAR \\
\\
\end{minipage}
\end{minipage}
\begin{minipage}{.74\textwidth}
\noindent \textbf{\color{bluePoli} Abstract:} {\abstract}
\end{minipage}
\end{center}

\vspace{12pt}

\newpage

\renewcommand*\contentsname{Table of Contents}
\tableofcontents

\newpage



%-----------------------------------------------------------------------------
%                               INTRODUCTION
%-----------------------------------------------------------------------------

\section{Introduction}

...



%-----------------------------------------------------------------------------
%                               DWA OVERVIEW
%-----------------------------------------------------------------------------

\section{DWA overview}

...



%-----------------------------------------------------------------------------
%                        DWA - IMPLEMENTATION IN ROS
%-----------------------------------------------------------------------------

\section{DWA - implementation in ROS}

\subsection{From ROS wiki}

...

\subsection{Comparison with the paper formulation}

...



%-----------------------------------------------------------------------------
%                             SETUP OF THE EXPERIMENT
%-----------------------------------------------------------------------------

\section{Setup of the experiment}

\subsection{The robot}

In our experiment we chose to simulate a small differential drive robot.\\

In particular, it is characterized by two main dimensions (specified as YAML parameters in the code):
\begin{itemize}
 \item \texttt{d} = 15 cm, which is the distance between the two motorized wheels;
 \item \texttt{r} = 3 cm, which is the radius of the two motorized wheels.\\
\end{itemize}

The precise footprint is a pentagon, just for convenience, so that when looking at it we are able to determine the orientation of the robot.
However, this is not a decisive detail since it has o influence on the robot's behavior.


\subsection{The map}

Regarding the map, it is important to highlight the different setting we have with respect to the usual DWA use.

There are two main differences:
\begin{itemize}
 \item in our setting there are no obstacles, neither fixed nor moving;
 \item the robot does not have any sensor.\\
\end{itemize}

As a consequence, we don't need both the local map and the global map, but only the local one.

Moreover, this map needs to be empy so it is generated strating from a totally white image.


\subsection{The trajectory}

The robot must follow a precise trajectory, which is used to perform all benchmarks.

In our case we chose an eight-shaped trajectory with a dimension of 2 x 1 meters.\\

In order to make DWA compute the velocities of the robot, we must feed it a goal.

This means that the complete trajectory has to be "discretized" in multiple points.
Each one of these points is passed to DWA as the current goal, and once it is reached the next point is set as the new goal. You will find a detailed explaination in the following sections.



%-----------------------------------------------------------------------------
%                               IMPLEMENTATION
%-----------------------------------------------------------------------------

\section{Implementation}

\subsection{Architecture overview}

... (things in common between our controller and DWA: simulator, ...)

\subsection{Simulator}

...

\subsection{Trajectory controller}

...

\subsection{DWA controller}

...



%-----------------------------------------------------------------------------
%
%-----------------------------------------------------------------------------

\section{Parameters tuning}

...



%-----------------------------------------------------------------------------
%
%-----------------------------------------------------------------------------

\section{Experimental Results}

... (plots of the bags + plots of the comparison with the custom script)



%-----------------------------------------------------------------------------
%
%-----------------------------------------------------------------------------

\section{Encountered problems}

...



%-----------------------------------------------------------------------------
%
%-----------------------------------------------------------------------------

\section{Usage of the code}

...



%-----------------------------------------------------------------------------
%                               CONCLUSIONS
%-----------------------------------------------------------------------------

\section{Conclusions}

...



\end{document}
